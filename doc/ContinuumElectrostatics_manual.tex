\documentclass[a4paper,11pt]{article}

% \usepackage{graphicx, overcite, supertabular, color, colordvi, setspace, amssymb, rotating}
\usepackage{xspace, hyperref, listings}

\pagestyle{plain}
\linespread{1.6}
\textwidth        16 cm
\textheight       22 cm
\oddsidemargin     0 mm
\evensidemargin    0 mm
\topmargin       -13 mm

% Prevent indenting paragraphs
\setlength\parindent{0pt}

\lstset{basicstyle=\ttfamily, frame=shadowbox, breaklines=true, breakatwhitespace=true}

\newcommand{\modulename}{ContinuumElectrostatics\xspace}
\newcommand{\pka}{$\mathrm{p}K_{\mathrm{a}}$\xspace}


%=================================================
\begin{document}

\begin{center}
{\LARGE \bf \modulename}

{\large
A pDynamo module for protonation state calculations

\vspace{1.0cm}
\underline{Mikolaj Feliks}\\
Institut de Biologie Structurale, Grenoble\\
Last updated: \today
}

\vspace{0.5cm}

\end{center}


%=================================================
\section{Introduction}
%=================================================
Proteins contain residues, cofactors and ligands that bind or release protons
depending on the current pH and the interactions with their molecular
environment.
%
These titratable residues, cofactors and ligands will be referred to as sites.
%
Each site exists in at least two charge forms, usually the protonated form and 
the deprotonated form.
%
These different charge forms are called instances.
%
The titration of proteins is difficult to study experimentally because
the available methods, such as calorimetry, cannot determine protonation states
of individual sites.
%
The knowledge of these individual protonation states is crucial for
understanding of many important processes, for example enzyme catalysis.


The \modulename module extends the pDynamo\cite{pDynamo_Field2008}
library with a Poisson-Boltz\-mann continuum electrostatic 
model that allows for calculations of protonation states of individual sites.
%
The module links pDynamo to the external solver of the Poisson-Boltzmann 
equation, MEAD.
%
MEAD is a program developed by Donald Bashford\cite{Bashford1997} and 
later extended by Timm Essigke\cite{Essigke_PhD} and Thomas Ullmann.
%
The electrostatic energy terms obtained with MEAD can be used to calculate
energies of all possible protonation states of the protein of interest.
%
However, analytic evaluation of protonation state energies is only possible
for proteins with usually less than 20 titratable sites.
%
The \modulename module provides an interface to the GMCT\cite{Ullmann2012}
program by Matthias Ullmann and Thomas Ullmann that can be used to sample
protonation state energies using a Monte Carlo method.
%
The GMCT interface makes it possible to study the titration of larger proteins.


%=================================================
\section{Copying}
%=================================================
The module is distributed under the CeCILL Free Software License, which is
a French equivalent of the GNU General Public License.
%
For details, see the files \texttt{Licence\_CeCILL\_V2-en.txt} (or
\texttt{Licence\_CeCILL\_V2-fr.txt} for the French version).


%=================================================
\section{Goals of the \modulename module}
%=================================================
I wrote the \modulename module primarily as a pretext to learn how
the Poisson-Boltzmann model works in detail.
%
I used this model very often during my studies on enzyme catalysis but never had
time nor will to learn the details of the theory that was behind.
%
I also wanted to better explore the pDynamo library, understand its programming
concepts and finally extend it with something useful.
%
Last but not least, I saw some room for improvement in the previously used
tools and scripts.

The \modulename module is similar in behavior to the \texttt{multiflex2qmpb.pl}
program, which is part of the QMPB package written by Timm Essigke.
%
The approach taken here is most compatible with the original approach
by Donald Bashford.
%
The key novelty is the improved treatment of multiprotic sites, 
such as histidine.
%
Also, calculations with the \makebox{\modulename} module are a lot faster than 
before because of the parallelization. 

% Talk about non-binary state vector and different energy reference (fully deprotonated system)
%
% For the calculated thermodynamic cycle, see Fig. 3.8, p. 82 in Timm's thesis.


%=================================================
\section{Installation and configuration}
%=================================================
Before the installation of the \modulename module, it is necessary
to have:
\begin{itemize}
  \setlength{\itemsep}{2pt}
  \item pDynamo 1.8.0
  \item Python 2.7
  \item GCC (any version should be fine)
  \item Extended MEAD 2.3.0
  \item GMCT 1.2.3
\end{itemize}
%
Extended MEAD and GMCT can be found on the website of Thomas Ullmann:

\url{http://www.bisb.uni-bayreuth.de/People/ullmannt/index.php?name=software}

Download the two packages and follow their respective installation
instructions.
%
The \modulename module requires for its functioning two programs
from the MEAD package, namely \texttt{my\_2diel\_solver} and \texttt{my\_3diel\_solver},
and the GMCT's main program, \texttt{gmct}.

\bigskip
In the next step, check out the latest source code of the module.
%
Note that for checking out the source code you should have Subversion installed
as well.

{\footnotesize \begin{lstlisting}
svn checkout http://pdynamo-extensions.googlecode.com/svn/trunk/ContinuumElectrostatics/
\end{lstlisting} }

\bigskip
Some parts of the module implementing the state vector are written in C and
therefore have to be compiled before use. 
%
In the future, I plan to shift some more parts of the module from Python to C.

\bigskip
Start from going to the directory \texttt{extensions/csource/}.
%
Change the uppermost line in the \texttt{Makefile}. 
%
This line defines the directory where you have installed pDynamo. 
%
After editing, close the file and run "make" to compile the C object file.

\bigskip
Go to the directory \texttt{extensions/pyrex/} and again edit the \texttt{Makefile}.
%
Change the line starting from "INC2" to the location of your pDynamo installation.
%
Close the file and run "make".
%
It should generate a dynamically linked library, \texttt{StateVector.so}, in 
the directory \texttt{Continuum\-Electrostatics/}.

\bigskip
At this point, the installation is complete.

\bigskip
Before using the module, the environment variable \texttt{PDYNAMO\_CONTINUUMELECTROSTATICS} 
should be set to the module's root directory.
%
This directory should be also added to the \texttt{PYTHONPATH} variable. 
%
This can be done in the following way (in Bash):

\newpage
{\footnotesize \begin{lstlisting}
export PDYNAMO_CONTINUUMELECTROSTATICS=/home/mikolaj/devel/ContinuumElectrostatics
export PYTHONPATH=$PYTHONPATH:$PDYNAMO_CONTINUUMELECTROSTATICS
\end{lstlisting} }


%=================================================
\section{Usage}
%=================================================
After the installation, it may be worth looking at some of the test cases.
%
I will explain the functioning of the module based on the test case "sites2".
%
This test uses a trivial polypeptide with only two titratable sites,
histidine and glutamate.
%
The test "histidine" uses only one site. The other tests use real-life,
although small proteins.


By default, the \modulename module will not empty the scratch directory
after completing the run.
%
Starting the job for the second time will cause reading of the existing 
scratch files instead of running the calculations anew.
%
To avoid this behavior, remove the scratch directory or declare another one
during the setup of the continuum electrostatic model.

\textbf{Warning! The energies calculated by the \modulename module are 
in units of kcal/mol, in contrast to kJ/mol, which are normally used 
in pDynamo.}


%-------------------------------------------------
\subsection{Setup of the protein model}
The electrostatic model used by the \modulename module requires that the protein
of interest is described by the CHARMM energy model\cite{MacKerell1998}.
%
In the first step, prepare CHARMM topology (PSF) and coordinate (CRD) files.
%
The preparation can be done using the programs CHARMM\cite{CHARMM_Brooks1983}
or VMD\cite{VMD1996}.
%
During the preparation of the protein model, all titratable residues in the protein
should be set to their standard protonation states at pH = 7, i.e. aspartates
and glutamates deprotonated, histidines doubly protonated, other residues
protonated.
%
The topology, coordinate and parameter files are loaded at the
beginning of the script \texttt{sites2.py}:

{\footnotesize \begin{lstlisting}
par_tab = ["charmm/toppar/par_all27_prot_na.inp", ]
mol  = CHARMMPSFFile_ToSystem ("charmm/testpeptide_xplor.psf", isXPLOR = True, parameters = CHARMMParameterFiles_ToParameters (par_tab))
mol.coordinates3 = CHARMMCRDFile_ToCoordinates3 ("charmm/testpeptide.crd")
\end{lstlisting} }


%-------------------------------------------------
\subsection{Setup of the continuum electrostatic model}
In the second step, a continuum electrostatic model is created:

{\footnotesize \begin{lstlisting}
ce_model = MEADModel (meadPath = "/home/mikolaj/local/bin/", gmctPath = "/home/mikolaj/local/bin/", scratch = "scratch", nthreads = 2)
\end{lstlisting} }

\bigskip
The parameter "meadPath" tells the directory where the MEAD programs,
\texttt{my\_2diel\_solver} and \texttt{my\_3diel\_solver}, are located.
%
The parameter "gmctPath" is the location of the \texttt{gmct} program.
%
If none of these directories are given, \texttt{/usr/bin} is assumed
by default.
%
The parameter "scratch" tells the directory where the MEAD job files
and output files will be written to.
%
If not present, this directory will be created.
%
The last parameter, "nthreads", defines the number of threads to be used.
%
By default \texttt{nthreads=1}, which means serial run.
%
Note that "nthreads" can be any natural number and that the calculations
scale linearly with the number of threads.
%
Parallelization is done at the coarse-grain level.
%
Since the electrostatic energy terms for a particular instance of a titratable
site can be calculated independently from energy terms of other instances of other
sites, each instance is assigned a separate thread.
%
A similar approach is taken during the calculations of titration curves,
where each pH-step of a curve is calculated separately.


\bigskip
In the next step, the continuum electrostatic model is initialized:

{\footnotesize \begin{lstlisting}
ce_model.Initialize (mol)
\end{lstlisting} }

\bigskip
The initialization means partitioning of the protein into titratable sites and
a non-titratable background.
%
It also means generating model compounds.
%
At this point, however, the input files for MEAD are not written and only
the necessary data structures inside the \texttt{MEADModel} object are created.
%
The \texttt{Initialize} method takes at least one argument,
which indicates the CHARMM-based protein model.

\bigskip
The next two lines generate a summary of the continuum electrostatic model and
write a table of titratable sites:

{\footnotesize \begin{lstlisting}
ce_model.Summary ()
ce_model.SummarySites ()
\end{lstlisting} }

\bigskip
After the model has been initialized, the input files necessary for calculations in MEAD
can be written to the scratch directory:

{\footnotesize \begin{lstlisting}
ce_model.WriteJobFiles (mol)
\end{lstlisting} }

By default, each site is assigned a separate directory, for example
\texttt{scratch/PRTA/GLU8}.
%
Inside the directory, there are PQR files for each instance of the site in the protein
and in a model compound.
%
The OGM and MGM files specify parameters of lattices used for solving the Poisson-Boltzmann equation.
%
The \texttt{back.pqr} file defines the non-titratable background.
%
The \texttt{protein.pqr} file defines the whole protein and is used to calculate the boundary
between the protein and the solvent.
%
The last file, \texttt{sites.fpt}, contains atomic coordinates and charges of all instances
and is used for calculating interaction energies between
different instances of sites in the protein.


%-------------------------------------------------
\subsection{Calculating electrostatic energies}
At this point, the electrostatic energy terms can be calculated:

{\footnotesize \begin{lstlisting}
ce_model.CalculateElectrostaticEnergies ()
\end{lstlisting} }

For each instance of each site, two electrostatic energy terms are calculated, namely
the Born energy ($G_{\mathrm{Born}}$) and the background energy ($G_{\mathrm{back}}$).
%
Born energy is the electrostatic energy of a set of charges interacting with its own
reaction field.
%
Background energy is the electrostatic energy of a set of charges interacting with
charges from outside of this set.
%
The two energies are calculated for a particular instance of a site both in the model compound
and in the protein.
%
The difference
$(G_{\mathrm{Born, protein}} + G_{\mathrm{back, protein}}) - (G_{\mathrm{Born, model}} + G_{\mathrm{back, model}})$
is calculated, which is called the homogeneous transfer energy, $G_{\mathrm{homotrans}}$.
%
Transferring of a site means moving it from the model compound to the protein.
%
In the model compound, the site has a model energy $G_{\mathrm{model}}$, which
corresponds to the experimentally known \pka value of the deprotonation reaction.
%
The site in the protein has an intrinsic energy $G_{\mathrm{intr}} = G_{\mathrm{model}} + G_{\mathrm{homotrans}}$.
%
The \texttt{my\_2diel\_solver} program calculates $G_{\mathrm{Born, model}}$ and $G_{\mathrm{back, model}}$.
%
The \texttt{my\_3diel\_solver} program calculates $G_{\mathrm{Born, protein}}$ and $G_{\mathrm{back, protein}}$
and, additionally, electrostatic interaction energies of an instance of a site
with other instances of other sites in the protein.
%
The \modulename module collects 
$G_{\mathrm{Born, protein}}$, $G_{\mathrm{back, protein}}$, 
$G_{\mathrm{Born, model}}$, $G_{\mathrm{back, model}}$ and interaction energies from MEAD 
and calculates $G_{\mathrm{homotrans}}$ and $G_{\mathrm{intr}}$ for each instance of each site.

Note that the \texttt{my\_3diel\_solver} program can in principle perform calculations
in a three-dielectric environment (solvent, protein, vacuum).
%
However, the model implemented in the \modulename module only deals with two-dielectric environments, 
i.e. the solvent phase and the protein/model compound phase.


%-------------------------------------------------
\subsection{Calculating microstate energies}

After the $G_{\mathrm{intr}}$ values and interaction energies have been calculated 
for all instances of all titratable sites, 
one can calculate the energy of a particular protonation state of the protein, i.e. 
the microstate energy, $G_{\mathrm{micro}}$.
%
The polypeptide in the "sites2" example contains only two sites, 
glutamate and histidine,
so there can be $2^1 * 4^1 = 8$ possible protonation states, because glutamate has two
instances ("p" and "d") and histidine has four instances ("HSP", "HSD", "HSE", "fully deprotonated").
%
For real-life proteins, the number of protonation states is very large.
%
The protonation state of the protein is defined by a state vector.
%
Each component of the state vector represents the protonation state of a site in the protein.
%
The value of the component indicates the current instance of the site.
%
The values of 0 and 1 do not necessarily mean "deprotonated" and "protonated".
%
The next few lines of code generate all possible permutations of the state vector and calculate
their respective microstate energies:

{\footnotesize \begin{lstlisting}
statevector = StateVector (ce_model)
increment   = True
while increment:
  Gmicro = ce_model.CalculateMicrostateEnergy (statevector, pH = 7.0)
  statevector.Print (ce_model, title = "Gmicro = %f" % Gmicro)
  increment = statevector.Increment ()
\end{lstlisting} }

\bigskip
In the lowest energy protonation state, the histidine is protonated at 
positions $\delta$ and $\epsilon$ and the glutamate is deprotonated.
%
Dissociation of the $\epsilon$-hydrogen from the histidine requires only 0.7\,kcal/mol.


%-------------------------------------------------
\subsection{Calculating protonation probabilities}
% For a given site, the probability of occurrence of a particular instance is calculated 
% from the Boltzmann weighted sum, which takes the form:
One option for calculating protonation probabilities is to use GMCT externally after 
the calculations by MEAD are complete.
%
This can be done by writing two files, \texttt{gintr.dat} and \texttt{W.dat}, containing 
the calculated intrinsic energies and interaction energies, respectively.

{\footnotesize \begin{lstlisting}
ce_model.WriteGintr ()
ce_model.WriteW ()
\end{lstlisting} }


\bigskip
In the \modulename module, protonation probabilities can be calculated either analytically for
small proteins or by using GMCT for larger proteins.
%
The following two lines calculate protonation probabilities analytically and report them
in a table.
%
By default, the calculations are done at pH = 7.
%
The default behavior can be changed by passing the "pH" argument. 

{\footnotesize \begin{lstlisting}
ce_model.CalculateProbabilitiesAnalytically ()
ce_model.SummaryProbabilities ()
\end{lstlisting} }


\bigskip
For comparison, the probabilities are calculated by using GMCT. 
%
During the calculations in GMCT, a \texttt{scratch/gmct/} directory will be created
for storing input and output files.

{\footnotesize \begin{lstlisting}
ce_model.CalculateProbabilitiesGMCT ()
ce_model.SummaryProbabilities ()
\end{lstlisting} }

\bigskip
The resulting tables show for each site the probability of occurrence of each instance. 
%
Instances with the highest probability of occurrence are marked with "*".
%
Optionally, the argument \texttt{reportOnlyUnusual=True} will only show sites 
in their non-standard  protonation states, for example protonated glutamates.


%-------------------------------------------------
\subsection{Calculating titration curves}
Titration curves are obtained by calculating protonation probabilities for a given pH-range, 
usually from 0 to 14.

{\footnotesize \begin{lstlisting}
ce_model.CalculateCurves (isAnalytic = True, forceSerial = True, directory = "curves_analytic")
\end{lstlisting} }

\bigskip
The argument \texttt{isAnalytic=True} tells that the protonation probabilities should be
calculated analytically.
%
Each pH-step will be calculated in a separate thread.
%
However, the analytic evaluation of titration curves seems to be very inefficient in parallel 
mode.
%
The argument \texttt{forceSerial=True} will switch on serial mode regardless of the number 
of threads declared previously.
%
For each instance of each site, a data file containing the titration curve will be generated.
%
The files will be placed in the directory specified by the argument \texttt{directory}.
%
The curves can be plotted in a plotting program, for example Gnuplot.

\bigskip
The same task can be accomplished by using GMCT in parallel mode:

{\footnotesize \begin{lstlisting}
ce_model.CalculateCurves (directory = "curves_gmct")
\end{lstlisting} }


%-------------------------------------------------
\section{Parameter files}
Parameter files in YAML format reside in the directory \texttt{parameters/}. 
%
The file \texttt{radii.yaml} defines a list of atomic radii required for calculating 
the boundary between the protein phase and the solvent phase.
%
The files in the \texttt{sites/} subdirectory define parameters for 
titratable sites.
%
The listing below shows the parameter file for aspartate:

\linespread{0.8}
{\footnotesize \begin{lstlisting}
---
site      : ASP
atoms     : [CB, HB1, HB2, CG, OD1, OD2]
instances :
  - label   : p
    Gmodel  : -5.487135
    protons : 1
    charges : [-0.21,  0.09,  0.09,  0.75, -0.36, -0.36]

  - label   : d
    Gmodel  : 0.000000
    protons : 0
    charges : [-0.28,  0.09,  0.09,  0.62, -0.76, -0.76]
...
\end{lstlisting} }
\linespread{1.6}


\bigskip
The site's name is defined by the parameter \texttt{site}. 
%
\texttt{atoms} defines a list of names of atoms that constitute the site.
%
The consecutive lines define instances of the site.
%
\texttt{label} is the name of the instance, for example "p" for "protonated".
%
\texttt{Gmodel} is the energy of the model compound expressed in kcal/mol.
%
$G_{\mathrm{model}}$ has been calculated at the temperature of 300\,K from the 
\pka value of aspartate, which is 4.0.
%
The values of model energies are recalculated if a different temperature is 
to be used.
%
The reason for $G_{\mathrm{model}} = 0$ for the deprotonated instance is that 
the energies are calculated relative to a reference state, which is a fully 
deprotonated state.
%
The parameter \texttt{protons} tells the number of protons bound to the site and 
is used during the calculation of the microstate energy.
%
Finally, \texttt{charges} is a list of atomic charges that differ between 
the instances.
%
The order of charges is the same as the order of atoms in the \texttt{atoms} list.
%
The charges usually come from the CHARMM force field, i.e. they are
parametrized for the dielectric constant $\epsilon = 4$.

Proteins often contain non-standard titratable sites, for example ligands or cofactors.
%
The parameter files describing these sites are automatically loaded from the current directory
as long as they have the extension "*.yaml".
%
If the names of the added sites coincide with the standard names, 
the parameters for the standard sites will be overwritten.


%=================================================
\section{Future developments}
%=================================================
\linespread{0.8}

\begin{itemize}
  \setlength{\itemsep}{1.0pt}
  \item Extend the model by including rotameric instances
  \item Include quantum chemical sites as in QMPB
  \item Include reduction/oxidation reactions in addition to protonation/deprotonation reactions
  \item Enclose the MEAD and GMCT codes inside pDynamo
  \item Optionally convert kcal/mol (MEAD units) to kJ/mol (pDynamo units)
  \item Move parts of the \texttt{WriteJobFiles} method to the instance class
  \item Improve the efficiency of writing job files
  \item Calculate ETA (Estimated Time for Accomplishment) during MEAD calculations
  \item Rename variables containing filenames so that they start from "file"
  \item Make use of the \texttt{pqr2SolvAccVol} program to speed up calculations 
  \item Use arrays instead of lists for storing interactions
  \item Move the method for calculating $G_{\mathrm{micro}}$ from Python to C
\end{itemize}

\linespread{1.6}


%=================================================
\section{References}
%=================================================
\renewcommand{\refname}{}
\vspace*{-1cm}

\linespread{0.8}
\bibliographystyle{ieeetr}
\bibliography{refs}

\linespread{1.6}


%-------------------------------------------------
\subsection*{Websites}

\begin{itemize}
  \item MEAD (Donald Bashford)\\ \url{http://stjuderesearch.org/site/lab/bashford/}

  \item Extended MEAD (Thomas Ullmann)\\ \url{http://www.bisb.uni-bayreuth.de/People/ullmannt/index.php?name=extended-mead}

  \item GMCT (Thomas Ullmann)\\ \url{http://www.bisb.uni-bayreuth.de/People/ullmannt/index.php?name=gmct-gcem}
\end{itemize}


% Doctoral thesis of Timm Essigke:\\
% \url{https://epub.uni-bayreuth.de/655/}


\end{document}
