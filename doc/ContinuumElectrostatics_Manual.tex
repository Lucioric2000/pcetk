\documentclass[a4paper,11pt]{article}

% \usepackage{graphicx, overcite, supertabular, color, colordvi, setspace, amssymb, rotating}
\usepackage{xspace, hyperref, listings}

\pagestyle{plain}
\linespread{1.6}
\textwidth        16 cm
\textheight       22 cm
\oddsidemargin     0 mm 
\evensidemargin    0 mm 
\topmargin       -13 mm

% Prevent indenting paragraphs
\setlength\parindent{0pt}

\lstset{basicstyle=\ttfamily, frame=shadowbox, breaklines=true, breakatwhitespace=true}

\newcommand{\modulename}{ContinuumElectrostatics\xspace}


%=================================================
\begin{document}

\begin{center}
{\LARGE \bf \modulename}

\vspace{0.5cm}
{\Large User Manual\\
Last update: \today}

\end{center}

\vspace{2cm}


%=================================================
\section{Introduction}
%=================================================
Proteins contain residues, cofactors and ligands that bind or release protons 
depending on the current pH and the interactions with their molecular 
environment.
%
These titratable residues, cofactors and ligands will be referred to as sites.
%
The titration of proteins is often difficult to study experimentally because 
the available methods, such as calorimetry, cannot determine protonation states 
of individual sites.
%
The knowledge of these individual protonation states is crucial for 
understanding of many important processes, for example enzyme catalysis.


The \modulename module extends the functionality of the pDynamo 
library with a Poisson-Boltzmann continuum electrostatic model that allows for 
calculations of protonation states of individual sites.
%
The module provides an interface between pDynamo and the external solver of 
the Poisson-Boltzmann equation, MEAD.
%
MEAD is a program developed by Donald Bashford and extended by Timm Essigke
and Thomas Ullmann.
%
The electrostatic energy terms obtained with MEAD can be used to calculate 
energies of all possible protonation states of the protein of interest.
%
However, analytic evaluation of protonation state energies is only possible 
for proteins with only a few titratable sites. 
%
The \modulename module also provides an interface to the GMCT 
program by Matthias Ullmann and Thomas Ullmann that can be used to sample 
protonation state energies using a Monte Carlo method. 
%
The GMCT interface allows for studying the titration of larger proteins.


In the present document, I focus on the practical side of the 
electrostatic calculations.
%
Other people have done a great work to develop the theory and computational 
methods the \modulename module is based on.
%
%The primary source of information was the doctoral thesis of Timm Essigke.


%=================================================
\section{Copying}
%=================================================
The module is distributed under the CeCILL Free Software License, which is
a French equivalent of the GNU General Public License.
%
For details, see the files \texttt{Licence\_CeCILL\_V2-en.txt} (or 
\texttt{Licence\_CeCILL\_V2-fr.txt} for the French version).


%=================================================
\section{Goals of the \modulename module}
%=================================================
I wrote the \modulename module primarily as a pretext to learn how 
the Poisson-Boltzmann model works in detail.
%
I used this model very often during my studies on enzyme catalysis but never had 
time nor will to learn the details of the theory that was behind.
%
I also wanted to better explore the pDynamo library, understand its programming
concepts and finally extend it with something useful.
%
Last but not least, I saw some room for improvement in the previously used 
tools and scripts. 

The \modulename module is similar in behaviour to the \texttt{multiflex2qmpb.pl} 
program, which is part of the QMPB package written by Timm Essigke.
%
The approach taken here is most compatibile with the original approach 
by Donald Bashford.
%
The key difference is that the treatment of multiprotic sites, such as histidine,
is improved.

% Talk about non-binary state vector and different energy reference (fully deprotonated system)

% 
% For the calculated thermodynamic cycle, see Fig. 3.8, p. 82 in Timm's thesis.


%=================================================
\section{Installation and configuration}
%=================================================
Before the installation of the \modulename module, it is necessary 
to have:
\begin{itemize}
  \setlength{\itemsep}{2pt}
  \item pDynamo 1.8.0
  \item Python 2.7
  \item GCC (any version should be fine)
  \item Extended MEAD 2.3.0
  \item GMCT 1.2.3
\end{itemize}
%
Extended MEAD and GMCT can be found on the website of Thomas Ullmann:

\url{http://www.bisb.uni-bayreuth.de/People/ullmannt/index.php?name=software}

Download the two packages and follow their respective installation 
instructions.
%
The \modulename module requires for its functioning two programs 
from the MEAD package, namely \texttt{my\_2diel\_solver} and \texttt{my\_3diel\_solver}, 
and the GMCT's main program, \texttt{gmct}.

\bigskip
In the next step, check out the latest source code of the module.
%
Note that for checking out the source code you should have Subversion installed 
as well.

{\footnotesize \begin{lstlisting}
svn checkout http://pdynamo-extensions.googlecode.com/svn/trunk/ContinuumElectrostatics/
\end{lstlisting} }

\bigskip
Some parts of the module implementing the state vector are written in C and
therefore have to be compiled before use. In the future, I plan to shift some 
more parts of the code from Python to C.

\bigskip
Go to the subdirectory \texttt{extensions/csource} and edit the Makefile. Only the
uppermost line has to be changed. It defines the directory where you have
installed pDynamo. After editing, close the file and type in \texttt{make} to compile
the C object file.

\bigskip
Go to the subdirectory \texttt{extensions/pyrex} and again edit the Makefile. Change
the line starting from "INC2" to the location of your pDynamo installation.
Close the file and type in \texttt{make}. It should generate a dynamically linked 
library \texttt{StateVector.so} in the \texttt{ContinuumElectrostatics} subdirectory.

\bigskip
At this point, the installation is complete.

\bigskip
Before using the module, you have to set the environment variable\\
\texttt{PDYNAMO\_CONTINUUMELECTROSTATICS} to the directory where you have checked out 
the source code. Also, you have to add this directory to the \texttt{PYTHONPATH}
variable. For example, you can do it this way (in Bash):

\newpage
{\footnotesize \begin{lstlisting}
export PDYNAMO_CONTINUUMELECTROSTATICS=/home/mikolaj/devel/ContinuumElectrostatics

export PYTHONPATH=$PYTHONPATH:$PDYNAMO_CONTINUUMELECTROSTATICS
\end{lstlisting} }

%=================================================
\section{Usage}
%=================================================
After the installation, it may be worth looking at some of the test cases.
%
I will explain the functioning of the module based on the test case "sites2".
%
This test uses a trivial polypeptide with only two titratable sites, 
histidine and glutamate. 
%
The test "histidine" uses only one site. The other tests use real-life, 
although small proteins.


%-------------------------------------------------
\subsection{Setup of the protein model}
The electrostatic model used by the \modulename module requires that the protein 
of interest is described by the CHARMM energy model.
%
In the first step, prepare CHARMM topology (psf) and coordinate (crd) files. 
%
The preperation can be done using the CHARMM program or VMD.
%
During the preparation of the protein model, all titratable residues in the protein 
should be set to their standard protonation states at pH = 7, i.e. aspartates 
and glutamates deprotonated, histidines doubly protonated, other residues 
protonated.
%
The topology, coordinate and parameter files are loaded at the
beginning of the script \texttt{sites2.py}:

{\footnotesize \begin{lstlisting}
par_tab = ["charmm/toppar/par_all27_prot_na.inp", ]

mol  = CHARMMPSFFile_ToSystem ("charmm/testpeptide_xplor.psf", isXPLOR = True, parameters = CHARMMParameterFiles_ToParameters (par_tab))

mol.coordinates3 = CHARMMCRDFile_ToCoordinates3 ("charmm/testpeptide.crd")
\end{lstlisting} }


%-------------------------------------------------
\subsection{Setup of the continuum electrostatic model}
In the second step, a continuum electrostatic model is created:

{\footnotesize \begin{lstlisting}
ce_model = MEADModel (meadPath = "/home/mikolaj/local/bin/", gmctPath = "/home/mikolaj/local/bin/", scratch = "scratch", nthreads = 2)
\end{lstlisting} }

\bigskip
The parameter "meadPath" tells the directory where the MEAD programs, 
\texttt{my\_2diel\_solver} and \texttt{my\_3diel\_solver}, are located.
%
The parameter "gmctPath" is the location of the \texttt{gmct} program.
%
If none of these directories are given, \texttt{/usr/bin} is assumed 
by default.
%
The parameter "scratch" tells the directory where the MEAD job files 
and output files will be written to.
%
If not present, this directory will be created.
%
The last parameter, "nthreads", defines the number of threads to be used.
%
By default \texttt{nthreads=1}, which means serial run.
%
Note that "nthreads" can be any natural number and that the calculations
scale linearly with the number of threads.
%
% The other parameters that can be set include:
% 
% \begin{itemize}
%   \item \texttt{temperature} - sets the temperature, the default is 300\,K
%   \item \texttt{ionicStrength} - sets the ionic strenght, the default is 0.1 
% \end{itemize}

\bigskip
In the next step, the continuum electrostatic model is initialized:

{\footnotesize \begin{lstlisting}
ce_model.Initialize (mol)
\end{lstlisting} }

\bigskip
The initialization means partitioning of the protein into titratable sites and
a non-titratable background.
%
It also means generating model compounds.
%
At this point, however, the input files for MEAD are not written and only 
the necessary data structures inside the MEADModel object are created.
%
The \texttt{Initialize} method takes at least one argument, 
which indicates the CHARMM-based protein model.

\bigskip
The next two lines print out summaries of the continuum electrostatic
model and the titratable sites:

{\footnotesize \begin{lstlisting}
ce_model.Summary ()
ce_model.SummarySites ()
\end{lstlisting} }

\bigskip
After the model has been initialized, the input files necessary for calculations in MEAD
can be written to the scratch directory:

{\footnotesize \begin{lstlisting}
ce_model.WriteJobFiles (mol)
\end{lstlisting} }

% ce_model.CalculateElectrostaticEnergies ()



%=================================================
\section{References}
%=================================================
MEAD website: \\
\url{http://stjuderesearch.org/site/lab/bashford/}

Extended MEAD website:\\ 
\url{http://www.bisb.uni-bayreuth.de/People/ullmannt/index.php?name=extended-mead}

Doctoral thesis of Timm Essigke:\\
\url{https://epub.uni-bayreuth.de/655/}


%=================================================
\section{Test cases}
%=================================================


%=================================================
\section{To-do list}
%=================================================
\begin{itemize}
  \setlength{\itemsep}{2pt}
  \item Move parts of WriteJobFiles to the instance class
  \item Rename variables containing filenames so that they start from "file"
  \item Coordinates from the FPT file should have their own data structure
  \item Efficiency improvements during writing job files
  \item Use arrays instead of lists for interactions (Real1DArray or SymmetricMatrix)
  \item Have a column of ETA (Estimated Time for Accomplishment) in MEAD calculations
  \item Optionally convert kcal/mol (MEAD units) to kJ/mol (pDynamo units)
  \item Make use of the pqr2SolvAccVol program to speed up the calculations a little bit
  \item The function calculating Gmicro should be written in C
\end{itemize}

\end{document}
