\documentclass[12pt]{article}

\usepackage{xspace, listings, setspace, geometry, xfrac}
\usepackage[super,sort&compress]{natbib}
\usepackage[plainpages=false,hyperfootnotes=false,linktocpage,bookmarks]{hyperref}

% The next three lines deal with the underscore character
\usepackage{lmodern}
\usepackage[T1]{fontenc}
\usepackage{textcomp}

% Page setup
\pagestyle{plain}
\linespread{1.6}
\textwidth       16 cm
\textheight      22 cm
\oddsidemargin    0 mm
\evensidemargin   0 mm
\topmargin      -13 mm

% Prevent indenting paragraphs
\setlength\parindent{0pt}

% Source codes
\lstset{basicstyle=\ttfamily, frame=shadowbox, breaklines=true, breakatwhitespace=true}

% New commands
\newcommand{\modulename}{\textit{Pcetk}\xspace}
\newcommand{\pka}{$\mathrm{p}K_{\mathrm{a}}$\xspace}
\newcommand{\kcal}{$\sfrac{\rm kcal}{\rm mol}$\xspace}
\newcommand{\kjoule}{$\sfrac{\rm kJ}{\rm mol}$\xspace}


%=================================================
\begin{document}

\begin{center}
{\LARGE \bf \modulename}

{\normalsize
% A pDynamo-based toolkit for protonation state calculations in proteins
A pDynamo-based continuum electrostatic toolkit

\vspace{1.0cm}
\underline{Mikolaj Feliks}\\
\makebox{\tt \href{mailto:mikolaj.feliks@gmail.com}{mikolaj.feliks@gmail.com}}\\
Institut de Biologie Structurale, Grenoble\\

Last updated: \today
}

\vspace{0.5cm}
\end{center}


%=================================================
\section{Introduction}
%=================================================
Proteins contain residues, cofactors and ligands that bind or release protons
depending on the current pH and the interactions with their molecular
environment\cite{Ullmann1999}.
%
These titratable residues, cofactors and ligands will be referred to as sites.
%
Each site exists in at least two charge forms, usually the protonated form and
deprotonated form.
%
Different charge forms of a site are called instances.
%
The titration of proteins is often difficult to study experimentally.
%
Moreover,
the available methods, such as calorimetry, cannot determine protonation states
of individual sites.
%
The knowledge of these individual protonation states is crucial for
understanding of many important processes, for example enzyme catalysis\cite{Bombarda2010}.


The \modulename toolkit extends the pDynamo library by
Martin Field\cite{pDynamo_Field2008}  with a Poisson-Boltz\-mann continuum electrostatic
model that allows for the calculation of protonation states
in proteins\cite{Bashford1992,Ullmann1999,Bombarda2006}.
%
The toolkit
connects pDynamo to the external solver of the Poisson-Boltzmann
equation, MEAD\cite{Bashford1997}.
%
The idea to employ MEAD for the calculation of electrostatic energy terms
is similar to how pDynamo interfaces with ORCA to calculate
quantum chemical energies and properties.
%

After the calculation of electrostatic energy terms by MEAD,
there are three ways to calculate the probabilities of protonation states.
%
First,
they can be calculated by using the statistical mechanical partition function,
which is, however, only possible for small proteins.
%
Second,
the probabilities can be estimated by using the in-house version of
the Metropolis Monte Carlo method\cite{Beroza1991}.
%
Third,
they can be estimated by the external Monte Carlo sampling program,
GMCT\cite{Ullmann2012,Thomas_PhD}.
%
The letter two methods make it possible to study the titration
of larger proteins.


%=================================================
\section{Copying}
%=================================================
The toolkit is distributed under the CeCILL Free Software License, which is
a French equivalent of the GNU General Public License.
%
More information can be found in files
{\tt Licence\_CeCILL\_V2-en.txt} or
{\tt Licence\_CeCILL\_V2-fr.txt} (the French version).


%=================================================
\section{Goals of the \modulename toolkit}
%=================================================
I wrote \modulename primarily as a pretext to learn how
the Poisson-Boltzmann model and Monte Carlo sampling work in detail.
%
I used this model very often during my PhD studies on enzyme catalysis but
never had time nor pressure to learn the details of the theory that was behind.
%
I also wanted to better explore pDynamo,
understand its programming concepts and finally extend it
with something useful.
%
Finally,
I saw some room for improvement in the previously used tools and scripts.

The \modulename toolkit aims at combining the functionalities of
earlier packages for studying the titration of proteins,
such as MEAD, QMPB and GMCT.
%
MEAD is a solver of the Poisson-Boltzmann equation,
written originally by Donald Bashford and extended by
Timm Essigke\cite{Essigke_PhD} and Thomas Ullmann\cite{Thomas_PhD}.
%
QMPB is a collection of Perl scripts for the preparation of
continuum electrostatic models of proteins,
written by Timm Essigke\cite{Essigke_PhD}.
%
GMCT is a Monte Carlo sampling program,
written by Matthias Ullmann
and extended by Thomas Ullmann\cite{Ullmann2012}.
%
In \modulename,
all these functionalities can be accessed by the user
in a convenient,
Python-based scripting environment.
%
The ``engine'' of \modulename that handles the actual calculation
is located in the underlying layer programmed in C.
%
The two layers are glued together with Cython.
%
Therefore,
the trade-off between the convenience of use of Python
and the speed of compiled languages is well balanced in the toolkit.


%=================================================
\section{Installation and configuration}
%=================================================
Before the installation of \modulename, it is necessary
to have:
\begin{itemize}
\itemsep0pt
  \item pDynamo 1.8.0
  \item Python 2.7 (including header files; python2.7-dev package in Debian)
  \item PyYAML 3.10 (python-yaml package in Debian)
  \item Cython 0.15.1 (\underline{optionally}; for recompiling Cython sources)
  \item GCC (any version should be fine)
  \item Extended-MEAD 2.3.0
  \item GMCT 1.2.3 (\underline{optionally})
\end{itemize}
%
Extended-MEAD and GMCT can be found on the website of Thomas Ullmann:

\url{http://www.bisb.uni-bayreuth.de/People/ullmannt/index.php?name=software}

Download the two packages and follow their respective installation
instructions.
%
The toolkit requires for its functioning two programs
from the Extended-MEAD package.
%
These are \texttt{my\_2diel\_solver} and \texttt{my\_3diel\_solver}.
%
In order to use GMCT for Monte Carlo sampling,
it is necessary to only have the
GMCT's main program, \texttt{gmct}.

\bigskip
In the next step,
download the latest source code of the toolkit from:

\url{http://github.com/mfx9/pcetk/archive/master.zip}

Alternatively,
clone the repository from GitHub:
%

{\footnotesize \singlespacing \begin{lstlisting}
git clone http://github.com/mfx9/pcetk.git
\end{lstlisting} }

\bigskip
Some parts of the toolkit's code are written in C or Cython and therefore have
to be compiled before use.
%
These parts include three submodules,
namely {\tt StateVector}, {\tt EnergyModel} and {\tt MCModelDefault}.

\bigskip
Go to directory \texttt{extensions/cython/}.
%
Edit the first line of \texttt{Makefile} starting from \makebox{``PDYNAMO\_PCORE''}.
%
The line should define the location of the pCore module of the present pDynamo installation.
%
After editing the file, run ``make''.
%
If you also want to recompile the Cython sources, precede ``make'' by running ``make clean\_all''.
%
If the Cython compilation fails,
it may be necessary to manually specify the location of the pCore module in
the {\texttt{cython\_compile.py} script.
%
Finally,
run ``make install'' to install the compiled library.

\bigskip
At this point, the installation is complete.

\bigskip
Before using the toolkit,
the environment variable \texttt{PDYNAMO\_PCETK}
should be assigned to point to the root directory of the toolkit.
%
This directory should also be added to the \texttt{PYTHONPATH} variable.
%
This can be done in the following way (in Bash):

{\footnotesize \singlespacing \begin{lstlisting}
export PDYNAMO_PCETK=/home/mikolaj/devel/pcetk
export PYTHONPATH=$PYTHONPATH:$PDYNAMO_PCETK
\end{lstlisting} }


%=================================================
\section{Usage}
%=================================================
After finishing the installation,
it may be worth looking at some of the test cases.
%
The functioning of the toolkit will be explained based on
the test case ``twosites''.
%
This test uses a trivial polypeptide with only two titratable sites,
histidine and glutamate.
%
The other tests use real-life,
although small proteins.


By default,
the scratch directory is not emptied after completing the run.
%
Starting the job for the second time will cause reading of the existing
scratch files instead of running the calculations anew.
%
To avoid this behavior,
remove the scratch directory or declare another one
during the setup of the continuum electrostatic model.

\textbf{Notice! The energies calculated by \modulename are
given in \kcal, in contrast to \kjoule, which are normally used
in pDynamo.}


%-------------------------------------------------
\subsection{Setup of the protein model}
The electrostatic model used by the toolkit requires that the protein
of interest is described by the CHARMM energy model\cite{MacKerell1998}.
%
In the first step,
prepare CHARMM topology (PSF) and coordinate (CRD) files.
%
The preparation can be done using,
for example,
CHARMM\cite{CHARMM_Brooks1983} or VMD\cite{VMD1996}.
%
During the preparation of the protein model,
all titratable residues in the protein
should be set to their standard protonation states at pH=7,
i.e.,
aspartates and glutamates should be deprotonated, histidines doubly protonated
and other residues protonated.
%
Topology, coordinate and parameter files are loaded at the
beginning of script \texttt{twosites.py}:

{\footnotesize \singlespacing \begin{lstlisting}
parameters = ["charmm/toppar/par_all27_prot_na.inp", ]
mol = CHARMMPSFFile_ToSystem ("charmm/testpeptide_xplor.psf", isXPLOR=True, parameters=CHARMMParameterFiles_ToParameters (parameters))
mol.coordinates3 = CHARMMCRDFile_ToCoordinates3 ("charmm/testpeptide.crd")
\end{lstlisting} }


%-------------------------------------------------
\subsection{Setup of the continuum electrostatic model}
%
In the second step, a continuum electrostatic model is constructed:

{\footnotesize \singlespacing \begin{lstlisting}
cem = MEADModel (system=mol, pathMEAD="/home/mikolaj/local/bin/", pathScratch="scratch", nthreads=2)
\end{lstlisting} }

\bigskip
%
The first parameter, {\tt system},
indicates the CHARMM-based protein model.
%
Parameter {\tt pathMEAD} specifies the directory where the MEAD programs,
{\tt my\_2diel\_solver} and {\tt my\_3diel\_solver}, are located.
%
If this directory is not given,
``/usr/local/bin'' is assumed
by default.
%
Parameter {\tt pathScratch} defines the directory where the MEAD job files
and output files will be written to.
%
If not present,
this directory will be created.
%
The last parameter,
{\tt nthreads},
defines the number of threads to use during parallel calculations.
%
By default {\tt nthreads=1}, which means serial run.
%
Note that {\tt nthreads} can be any natural number and that the calculations
scale linearly with the number of threads.
%
Parallelization is done at the coarse-grain level.
%
Since the electrostatic energy terms for a particular instance of a titratable
site can be calculated independently from energy terms of other instances of
sites, each instance is assigned a separate calculation thread.
%
A similar approach is taken during the calculation of titration curves,
where each pH-step of a curve is calculated separately.


\bigskip
In the next step,
the continuum electrostatic model is initialized:

{\footnotesize \singlespacing \begin{lstlisting}
cem.Initialize ()
\end{lstlisting} }

\bigskip
During the initialization,
the protein model is partitioned into titratable sites and
a non-titratable background.
%
Additionally,
model compounds are generated.
%
At this point, however, the input files for MEAD are not written and only
the necessary data structures inside the \texttt{MEADModel} object are created.

\bigskip
The next two lines generate a summary of the continuum electrostatic model and
show a table of titratable sites:

{\footnotesize \singlespacing \begin{lstlisting}
cem.Summary ()
cem.SummarySites ()
\end{lstlisting} }

\bigskip
After the model has been initialized,
the input files necessary for calculations in MEAD can be written to
the scratch directory:

{\footnotesize \singlespacing \begin{lstlisting}
cem.WriteJobFiles ()
\end{lstlisting} }

By default,
each site is assigned a separate directory,
for example {\tt scratch/PRTA/GLU8/}.
%
Inside,
there are PQR files for each instance of the site in the protein
and in a model compound.
%
The OGM and MGM files specify parameters of lattices used to solve the Poisson-Boltzmann equation.
%
File \texttt{back.pqr} defines the non-titratable background
and \texttt{protein.pqr} defines the whole protein and is used to calculate the boundary
between the protein and the solvent.
%
The boundary is calculated automatically by MEAD.
%
The last file,
\texttt{sites.fpt},
contains atomic coordinates and charges of all instances
and is used for the calculation of interaction energies between
different instances of sites in the protein.


%-------------------------------------------------
\subsection{Calculating electrostatic energy terms}
At this point, the electrostatic energy terms can be calculated:

{\footnotesize \singlespacing \begin{lstlisting}
cem.CalculateElectrostaticEnergies ()
\end{lstlisting} }

For each instance of each site,
two electrostatic energy terms are calculated,
namely
the Born energy, $G^{\rm Born}$, and
the background energy, $G^{\rm back}$.
%
Born energy is the electrostatic energy of a set of charges interacting with its own
reaction field.
%
Background energy is the electrostatic energy of a set of charges interacting with
other charges from outside of this set.
%
The two energies are calculated for a particular instance of a site
both in the model compound and in the protein.
%
The difference
$\left( G^{\rm Born}_{protein} + G^{\rm back}_{protein} \right) - \left( G^{\rm Born}_{model} + G^{\rm back}_{model} \right)$
is obtained,
which is called the heterogeneous transfer energy, $G^{\rm heterotrans}$.
%
The site is transferred from the model compound to the protein.
%
In the model compound,
the site has a model energy $G^{\rm model}$,
which corresponds to the experimentally known \pka value of the deprotonation reaction in aqueous solution.
%
The site in the protein has an intrinsic energy,
$G^{\rm intr} = G^{\rm model} + G^{\rm heterotrans}$.
%
Program \texttt{my\_2diel\_solver} calculates Born and background energies in the model compound.
%
Program \texttt{my\_3diel\_solver} calculates Born and background energies in the protein and,
additionally,
electrostatic interaction energies of the instance
with all other instances of other sites in the protein.
%
The toolkit collects
Born,
background
and interaction energies from MEAD
and calculates transfer and intrinsic energies.


Note that \texttt{my\_3diel\_solver} can in principle perform calculations
in a three-dielectric environment (solvent, protein, vacuum).
%
However,
the electrostatic model employed here uses only two-dielectric environments,
namely the solvent phase and protein/model compound phase.


%-------------------------------------------------
\subsection{Calculating microstate energies}

After the $G^{\rm intr}$ values and interaction energies have been calculated and tabulated,
one can calculate the energy of a particular protonation state of the protein,
which is the so-called microstate energy,
$G^{\rm micro}$.
%
The polypeptide in the ``twosites'' example contains only two sites,
glutamate and histidine,
so there can be $2^1 * 4^1 = 8$ possible protonation states,
because glutamate has two
instances (``p'' and ``d'')
and histidine has four instances (``HSP'', ``HSD'', ``HSE'', ``fully deprotonated'').
%
For real-life proteins, the number of protonation states is very large.
%
The protonation state of a protein is defined by a state vector.
%
Each component of the state vector represents the protonation state of a site in the protein.
%
The value of the component indicates the current instance of the site.
%
The values of 0 and 1 do not necessarily mean ``deprotonated'' and ``protonated''.
%
The next few lines of code generate all possible permutations of the state vector and calculate
their respective microstate energies:

{\footnotesize \singlespacing \begin{lstlisting}
statevector = StateVector (cem)
increment   = True
while increment:
    Gmicro = cem.CalculateMicrostateEnergy (statevector, pH=7.0)
    statevector.Print (verbose=True, title="Gmicro = %f" % Gmicro)
    increment = statevector.Increment ()
\end{lstlisting} }


%-------------------------------------------------
\subsection{Calculating probabilities of protonation states}
In the \modulename toolkit,
the probabilities of protonation states
can be calculated either analytically for small proteins
or by Monte Carlo sampling for larger proteins.
%
If a Monte Carlo model has not been defined,
the default option is to calculate the probabilities analytically:

{\footnotesize \singlespacing \begin{lstlisting}
cem.CalculateProbabilities (pH=7)
cem.SummaryProbabilities ()
\end{lstlisting} }

\bigskip
Argument {\tt pH} can be omitted,
since pH=7 is the default setting.
%
For comparison,
the probabilities can be calculated by using a Monte Carlo method.
%
In the first step,
a MC model is created and a number of 30000 production scans is defined.
%
In the second step,
the MC model is associated with the continuum electrostatic model
by using the {\tt DefineMCModel} method.
%
From this moment on,
the probabilities are estimated with Metropolis Monte Carlo.

{\footnotesize \singlespacing \begin{lstlisting}
mc = MCModelDefault (nprod=30000)
cem.DefineMCModel (mc)
cem.CalculateProbabilities ()
cem.SummaryProbabilities ()
\end{lstlisting} }

\bigskip
The resulting tables show for each site the probability of occurrence of each instance.
%
Instances with the highest probability of occurrence are marked with ``*''.
%
Optionally, argument {\tt reportOnlyUnusual=True} will only show sites
in their non-standard  protonation states, for example protonated glutamates.


%-------------------------------------------------
\subsection{Calculating titration curves}
Titration curves are obtained by calculating protonation state probabilities
for a given pH-range,
usually from 0 to 14.
%
The resolution of a curve can be adjusted with
the {\tt curveSampling} option (default 0.5 pH-unit).

{\footnotesize \singlespacing \begin{lstlisting}
from ContinuumElectrostatics import TitrationCurves
cmc = TitrationCurves (cem, curveSampling=0.5)
cmc.CalculateCurves ()
cmc.WriteCurves (directory="curves_mc")

cem.DefineMCModel (None)
ca = TitrationCurves (cem, curveSampling=0.5)
ca.CalculateCurves ()
ca.WriteCurves (directory="curves_analytic")
\end{lstlisting} }

\bigskip

A {\tt cmc} object is created representing titration curves.
%
The curves are subsequently calculated and the results are written to files
in {\tt curves\_mc/} directory.
%
A simple two-column text file is written for each instance of each site that
can be visualized in a plotting program of choice.
%
Next,
the Monte Carlo model is detached from the electrostatic model by passing
a {\tt None} argument.
%
The curves are recalculated analytically and written to directory
{\tt curves\_analytic/}.


%-------------------------------------------------
\subsection{Calculating microstate energies of substates}
From the calculated probabilities,
one can generate a state vector describing
the lowest energy protonation state of the protein:

{\footnotesize \singlespacing \begin{lstlisting}
lowestEnergyVector = StateVector_FromProbabilities (cem)
\end{lstlisting} }

\bigskip
The microstate energy ($G_{\mathrm{micro}}$) in the lowest energy protonation state can
be calculated in the following way:

{\footnotesize \singlespacing \begin{lstlisting}
Gmicro = cem.CalculateMicrostateEnergy (lowestEnergyVector)
\end{lstlisting} }

\bigskip
By changing the values (=instances) of selected components (=sites) of the state vector,
it is possible to calculate energies for a series of substates of
the lowest energy state.
%
Alternatively,
substate energies can be calculated easier with the \texttt{MEADSubstate} class:

{\footnotesize \singlespacing \begin{lstlisting}
sites = (
    ("PRTA", "GLU" , 35),
    ("PRTA", "ASP" , 52), )
substate = MEADSubstate (cem, sites, pH=7.0)
substate.CalculateSubstateEnergies ()
substate.Summary ()
\end{lstlisting} }

\bigskip
The above code was taken from the test case ``lysozyme''.
%
First,
a tuple of sites that make up the substate is defined.
%
Second,
a substate object is created and the lowest energy state vector is determined
automatically by the toolkit at pH=7.
%
Finally,
state energies are calculated for the substate and a summarizing table is
printed:

{\footnotesize \singlespacing \begin{lstlisting}
-----------------------------------------------------------
 State  Gmicro  Charge  Protons  PRTA GLU 35   PRTA ASP 52
-----------------------------------------------------------
     1     0.00      -2       0      d             d
     2     2.16      -1       1      p             d
     3     4.26      -1       1      d             p
     4     7.69       0       2      p             p
-----------------------------------------------------------
\end{lstlisting} }

\bigskip
In the lowest energy state,
Glu35 and Asp52 are deprotonated
and the protonation of Glu35 by a proton coming from the solution
requires about 2.2 \kcal.


%-------------------------------------------------
\section{Parameter files}
Two parameter formats are recognized,
YAML (pDynamo's default format) and
EST (an extension of the ST format used by MEAD).
%
File {\tt parameters/radii.yaml} contains a list of atomic radii
required to calculate the boundary between the protein and
solvent phase.
%
Files in the {\tt parameters/sites/} directory define parameters
for titratable sites.
%
The listing below shows the parameter file for aspartate:

\newpage
{\footnotesize \singlespacing \begin{lstlisting}
---
site      : ASP
atoms     : [CB, HB1, HB2, CG, OD1, OD2]
instances :
  - label   : p
    Gmodel  : -5.487135
    protons : 1
    charges : [-0.21,  0.09,  0.09,  0.75, -0.36, -0.36]

  - label   : d
    Gmodel  : 0.000000
    protons : 0
    charges : [-0.28,  0.09,  0.09,  0.62, -0.76, -0.76]
...
\end{lstlisting} }


\bigskip
The site's name is defined by the parameter \texttt{site}.
%
\texttt{atoms} defines a list of names of atoms that constitute the site.
%
The consecutive lines define instances of the site.
%
\texttt{label} is the name of an instance, for example ``p'' for ``protonated''.
%
\texttt{Gmodel} is the energy of the model compound expressed in \kcal.
%
The $G^{\rm model}$ value is derived from the aqueous solution \pka value
of aspartate,
which is 4.0,
calculated at 300 K.
%
Model energies are recalculated if a different temperature is
to be used.
%
The reason for $G^{\rm model}=0$ for the deprotonated instance is that
the energies are calculated relative to a reference state,
which is in this case deprotonated aspartate.
%
Parameter {\tt protons} defines a number of protons bound to
each protonation form (=instance) of aspartate.
%
Finally,
{\tt charges} is a list of atomic charges that differ between
the instances.
%
The order of charges is the same as the order of atoms in
the {\tt atoms} list.
%
The charges usually come from the CHARMM force field,
i.e.,
they are parametrized for the dielectric constant $\epsilon=4$.

Proteins often contain non-standard titratable sites,
for example ligands or cofactors.
%
Parameter files describing these sites are recognized by their extension
and can be loaded automatically from the current directory.
%
These additional parameter files can be provided both in the EST or YAML formats
(the EST format takes precedence).
%
If the names of the new sites coincide with the names of already existing sites,
the parameters of the existing sites will be overwritten.


%=================================================
\section{Future developments}
%=================================================
\begin{itemize}
\itemsep0pt
  \item Extend the model by including rotameric instances
  \item Include reduction/oxidation reactions in addition to protonation/deprotonation reactions
  \item Develop a custom Poisson-Boltzmann solver to replace MEAD
  \item Optionally convert \kcal (MEAD units) to \kjoule (pDynamo units)
  \item Make use of the MEAD program {\tt pqr2SolvAccVol} to speed up calculations a bit
\end{itemize}


%=================================================
\section{Feedback}
%=================================================
The \modulename toolkit and its documentation
are in a stage of active development.
%
Therefore,
bugs are inevitable.
%
If you found a bug,
have a question or wish contribute to the code,
please write to
the developer:
\makebox{\tt \href{mailto:mikolaj.feliks@gmail.com}{mikolaj.feliks@gmail.com}}.


%=================================================
\section{References}
%=================================================
\renewcommand{\refname}{}
\vspace*{-1cm}

\bibliographystyle{abbrv}
\begingroup
    {\small
        \setlength{\bibsep}{8pt}
        \setstretch{1}
        \bibliography{refs}
    }
\endgroup


%-------------------------------------------------
\subsection*{Websites}

\begin{itemize}
\itemsep0em
  \item pDynamo (Martin J. Field)\\ \url{http://pdynamo.org}
  \item MEAD (Donald Bashford)\\ \url{http://stjuderesearch.org/site/lab/bashford/}
  \item Extended-MEAD (Thomas Ullmann)\\ \url{http://www.bisb.uni-bayreuth.de/People/ullmannt/index.php?name=extended-mead}
  \item GMCT (Thomas Ullmann)\\ \url{http://www.bisb.uni-bayreuth.de/People/ullmannt/index.php?name=gmct-gcem}
\end{itemize}


\end{document}
